\documentclass{article}
\usepackage[utf8]{inputenc}
\usepackage{mathtools}
\usepackage{physics}
\renewcommand{\vec}[1]{\mathbf{#1}}
\title{Advanced solid state physics exam}
\author{Rasmus Hansen, Asbjørn Moltke and Victor Elkjaer}
\date{May 2018}

\begin{document}
\maketitle

\section{Static and dynamic screening in metals}
Consider an external field. From linear response the material responds at the same frequency, giving an induced density
\begin{equation}
    n_{ind}(r) = -q u_{tot} = -q(A_{tot}e^{iqr}+C.C)
\end{equation}
The internal field is the total minus the induced, giving a dielectric function given by
\begin{equation}
    \varepsilon = \frac{q^2}{q^2+\rho(E_F)}
\end{equation}
Inserting this in the point charge potential in Fourier space $1/r = \int 1/q^2 \exp(i\mathbf{q} \cdot \mathbf{r}) \mathrm{d} \mathbf{q}$ gives screened potential
\begin{equation}
   U= \frac{1}{r} \exp(- k_{TF} r)
\end{equation}
$k_{TF} \approx 0.5 Å$. This model has problems. Induced charge diverges at zero, and no Friedel oscillations. Instead turn to quantum mechanical expression.
\begin{equation}
    \varepsilon = 1 + \frac{8 \pi}{q^2 V} \sum_{\alpha \beta} \frac{\abs{\mel{\alpha}{e^{i\mathbf{q} \cdot \mathbf{r}}}{\beta}}^2}{E_{\beta}-E_{\alpha}-\omega - i \eta}(f_{\alpha}-f_{\beta})
\end{equation}
Assume jellium/free electron gas and plug in plane waves. In the static limit this leads to Lindhart
\begin{equation}
    \varepsilon(\omega=0,q) = 1 + \frac{k_{TF}}{q^2}F(q/2k_{TF})
\end{equation}

The induced charge can in general be found as
\begin{equation}
    \rho_{ind}(\mathbf{r}) = Z e \frac{1}{(2 \pi)^3} \int \left[ \frac{1}{\varepsilon(q)}-1 \right] e^{i \mathbf{q} \cdot \mathbf{r}} \mathrm{d} \mathbf{q}
\end{equation}
This leads to two finite induced charge and Friedel.



\newpage
\section{Plasmons}
\subsection{Plasmon relation to dielectric function}
Plasmons are clearly seen to exist when the dielectric function (of $\omega$) goes to zero $\rightarrow$ induces a finite total field regardless of the size of the external.
\begin{equation}
    v_{tot}(r,\omega) = \int\epsilon^{-1}(r,r',\omega)v_{ext}(r',\omega)\mathrm{d}r'
\end{equation}
Single e-h excitations vs. collective excitation (Sketch tangent function)\\
Equation of motion technique:
\begin{equation}
    [\hat{S}_i,\hat{H}] = (E_i - E_0)\hat{S}_i \quad, \; \hat{S}(q) = \dfrac{1}{\sqrt{N}}\sum_{k}\phi_k(q)\hat{S}_k(q)
\end{equation}

RPA $\rightarrow$ a maximum of one excitation per excited state.\\
Landau damping, sketch for simple metals.
\subsection{Plasmon energy dispersion}
Starting from the Lindhart dielectric function. Upon solving and Taylor expanding, it is obtained
\begin{equation}
    \varepsilon(q \rightarrow 0,\omega) = 1 - \frac{\omega_p^2}{(\omega + i \eta)^2} - \frac{3}{5} \frac{\omega_p^2}{(\omega+i\eta)^2}v_F^2 q^2
\end{equation}
\begin{equation}
    \omega_{pl}(q) = \omega_p\left(1 + \dfrac{3 v_f^2q^2}{10\omega_p^2} + \ldots\right)
\end{equation}

\subsection{Surface plasmon-polaritons}
\textbf{Qualitative features:} couples with light. Is not self-sustained. Requires momentum transfer from e.g. umklapp.\\ \textbf{Thin films:} Plasmon-polaritons at both surfaces.


\newpage
\section{Linear Reponse}
\subsection{Kubo formula}
Small perturbation $\Rightarrow$ linear response
Interaction picture, where time evolution of operators is in Heisenberg
\begin{equation}
    \hat{A}_{H_0}(t) = e^{i \hat{H_0} t} \hat{A}(t) e^{-i \hat{H_0} t}
\end{equation}
While the time dependence of due to the perturbation is carried in the wave functions in the Scrödinger picture
\begin{equation}
    \ket{\psi(t)} = U_I \ket{\psi(t_0)}, \quad U_I(t,t_0) = e^{i \hat{H_0} t} U e^{-i \hat{H_0} t_0}
\end{equation}
Inserting in Schrödinger one can obtain, after taylor expanding to first order, and moving to non interacting picture $\hat{U} \approx \hat{T}(1-i\int_{t_0}^0 \hat{V}_{\hat{H}_0}(t')\mathrm{d}t') e^{i \hat{H}_0 t_0}$. \\
A change in the expectation value of any time independent (assumption) observable to first order is
\begin{equation}
    \delta \hat{A} = \mel{0}{\hat{A}}{0} - \mel{0}{U^{\dagger}(0)\hat{A}U(0)}{0}
\end{equation}
Leading to Kubo formula
\begin{equation}
    \delta A(t=0) = -i \int_{t_0}^{\infty} \theta(-t') \mel{0}{\comm{\hat{A}_{\hat{H}_0}(0)}{\hat{V}_{\hat{H}_0}(t')}}{0} \mathrm{d} t'
\end{equation}
\subsection{time varying pertubation}
Assuming $\hat{V}(t) = e^{-i(\omega + i \eta)t} \hat{V}$, one can obtain
\begin{equation}
    \delta A(\omega + i \eta) = \sum_s \frac{\mel{0}{\hat{A}}{s} \mel{s}{\hat{V}}{0}}{\omega - \omega_{s0} + \i \eta} -  \frac{\mel{0}{\hat{V}}{s} \mel{s}{\hat{A}}{0}}{\omega + \omega_{s0} + \i \eta}
\end{equation}
\subsection{Non interacting Density-density response}
Important case, related to the dielectric function. The observable is now the density $\hat{A} = \hat{n}(r)$, and the perturbation is still adiabatic and time varying, and given additionally given by $\hat{V} = \int V(r) \hat{n}(r) \mathrm{d}r$. 
\begin{equation}
    \delta n(r, \omega) = \int \chi(r,r',\omega) V(r') \mathrm{d}r'
\end{equation}
Where
\begin{equation}
    \chi(r,r',\omega) = \sum_s \frac{\mel{0}{\hat{n}(r)}{s} \mel{s}{\hat{n}(r')}{0} }{\omega - \omega_{s0} + \i \eta} - \frac{\mel{0}{\hat{n}(r)}{s} \mel{s}{\hat{n}(r')}{0}}{\omega + \omega_{s0} + \i \eta}
\end{equation}

Non-interacting means we can introduce the density operator $\hat{n}(r) =  \sum_i \phi_i^* \phi_i \hat{c}^{\dagger}_i \hat{c}_i$

Arrive at:
\begin{equation}
    \chi(r,r',\omega) = \sum_{ik} (f_i-f_j)\dfrac{\psi_i(r)^{*}\psi_j(r)\psi_i(r')\psi_j(r')^{*}}{\omega-(\varepsilon_j-\varepsilon_i)+i\eta}
\end{equation}



\newpage
\section{Density response function}
The change in the total potential can be given in two ways
\begin{equation}\label{eq:density_response_dielectric}
    \delta v_{tot}(r,\omega) = v_{ext}(r,\omega) + \int \frac{\delta n(r_1,\omega)}{\abs{r-r_1}} \mathrm{d}r_1 = \int \varepsilon^{-1} v_{ext}(r_1,\omega) \mathrm{d}r_1
\end{equation}
This defines the dielectric function. Take the functional derivative with $v_{ext}(r',\omega)$. The definition of the density-density response function is
\begin{equation}
    \chi(r_1,r',\omega) = \frac{\delta n(r_1,\omega)}{\delta v_{ext} (r',\omega)}
\end{equation}
Providing
\begin{equation}
    \varepsilon^{-1}(r,r',\omega) = \delta(r-r') + \int \frac{1}{\abs{r-r_1}} \chi(r_1,r',\omega) \mathrm{d}r_1
\end{equation}

\subsection{Dielectric function within RPA}
Again, start from \eqref{eq:density_response_dielectric} take functional derivative with respect to $v_{ext}$\\
Use
\begin{equation}
    \delta v_{tot} = \delta v_{ext} + \int \dfrac{n(r',\omega)}{\abs{r-r'}}\mathrm{d}r \; \Rightarrow \; \dfrac{\delta v_{ext}}{\delta v_{tot}} = \delta(r-r') - \int \dfrac{\delta n(r_1,\omega)}{\delta v_{tot}(r)}\dfrac{1}{\abs{r-r_1}}\mathrm{d}r_1
\end{equation}
This is equal to $\epsilon(r,r',\omega)$. Identify $\chi^{0}(r_1,r',\omega)$.


\subsection{Local Field effects}
Go from
\begin{equation}
    \epsilon(r,r',\omega) \quad (\mathrm{FT})\Rightarrow \quad \epsilon_{G,G'}(q,\omega)
\end{equation}
Talk about macroscopic dielectric constant: \textbf{Think of discrete dielectric function (Matrix)}
\begin{equation}
    \mathrm{RIGHT:} \;\epsilon_{M}(\omega) = \lim_{q\rightarrow 0}\dfrac{1}{\epsilon^{-1}_{00}(q,\omega)} \quad ,\mathrm{Wrong:} \; \epsilon_{M} = \lim_{q\rightarrow0}\epsilon_{00}(q,\omega)
\end{equation}

\newpage
\section{Excitons}
\subsection{Joint density of states and interband transitions}
\begin{itemize}
    \item Difference from plasmons (Plasmon at higher energies, Exciton lowers energy).
    \item Semiconductors and insulators, interband transitions, BSE not RPA. exchange (requires overlap) vs. Coulomb ("requires" not too much screening).
    \item Difference from trivial excitations (Bound state).
    \item Joint density of states and critical points
\end{itemize}
  

\begin{equation}
    \text{Critical Points at: } \nabla_k \qty(E_{ck} - E_{vk}) = 0
\end{equation}
\subsection{Two models: Simple two band and screened hydrogen model}
Simple two band model, tight binding, localised orbitals.
\begin{equation}
    States: \ket{\Phi_{n,q=0}} = \dfrac{1}{\sqrt{N}}\sum_{m=0}^{N-1}b_{n+m,c}^{\dagger}b_{m,v}
    ^{\dagger}\ket{\Psi_0}
\end{equation}
\begin{equation}
    H_0 = E_0 -\left(\sum \varepsilon_v b^{\dagger}_{nv}b_{nv} + t(b^{\dagger}_{nv}b_{n+1,v} + b^{\dagger}_{nv}b_{n-1,v}) - electron\right)
\end{equation}
\begin{equation}
    H_{int} = - \sum_{m,m}^{N-1}\dfrac{U}{1+\abs{n-m}}b^{\dagger}_{nv}b_{nv}b^{\dagger}_{mv}b_{mv}
\end{equation}
Solve: $\mathbf{H}\mathbf{F}_i = E_i \mathbf{F}_i$.\\
The screened hydrogen model takes its offset in an expansion of single particle excitation functions.
\begin{equation}
    \Psi_{ex} = \sum_k A(k) \Phi_{c \mathbf{k}+\mathbf{k}_{ex},vk}
\end{equation}
Using the vanishing momentum of light at optical frequencies, and assuming a two band parabolic model. One can obtain a hydrogen like equation for the envelope function
\begin{equation}
    F(\mathbf{r}) = \frac{1}{\sqrt{V}} \sum_{\mathbf{k}} A(\mathbf{k}) e^{i \mathbf{k} \cdot \mathbf{r}}
\end{equation}
Where $\mathbf{r}$ describes the distance between electron and hole.
The hydrogen like equation:
\begin{equation}
    \qty[-\frac{\hbar^2 \nabla^2}{2 \mu_{ex}}-\frac{e^2}{\varepsilon r}] F(\mathbf{r}) = (E-E_G)F(\mathbf{r})
\end{equation}
where $F\qty(r)$ denotes the envelope function which physical meaning represents the position of the electron, given the hole is at the origin ($r=0)$. Typical exciton binding energies are meV for typical semiconductors, and eV for strong insulators.

\subsection{Role of the effective exciton mass:}
\textbf{Flat bands} result in a high exciton mass and a localised exciton whereas \textbf{dispersive bands} result in small exciton mass and a delocalised exciton. The exciton mass is given as $\mu_{\mathrm{ex}}^{-1} = m_{\mathrm{e}}^{-1} + m_{\mathrm{h}}^{-1}$. Recall that the inverse of the effective electron and hole masses can be calculated as the curvature of the bands at $k=0$ for the conduction and valence band (electrons and holes respectively).

\newpage
\section{Green functions and quasiparticles}
\begin{itemize}
    \item What is a quasiparticles?
    \item What is a Green function? 
    \begin{equation}
        G(x,x') = - \theta(t-t') \mel{N}{\acomm{\Psi(x)}{\Psi^{\dagger}(x')}}{N}, \quad \Psi(x) = e^{-i H t} \Psi(r) e^{i H t}
    \end{equation}
    \item Fourier transforming the Greens function leads to
\begin{equation}
    G(r,r';\omega) = \sum_i \dfrac{\Psi_{i+}^{QP}(r)\Psi_{i+}^{QP}(r')^{*}}{\omega - \varepsilon_{i+}^{QP} + i\eta} + \sum_i \dfrac{\Psi_{i-}^{QP}(r)\Psi_{i-}^{QP}(r')^{*}}{\omega - \varepsilon_{i-}^{QP} + i\eta}
\end{equation}
$\Psi_{i+}^{QP}(r)$ are quasiparticle wavefunctions.
\item Spectral properties: spectral function imaginary part of Green's functions
\item Projected Green's function $G_{aa}(\omega) = G^{0}_{aa}(\omega) + G^{0}_{aa}(\omega)\sum_{k}V_{ak}G_{ak}(\omega)$
\begin{equation}
    [(\omega + i\eta)I -H_0]G^{0}(\omega) = I \quad, [(\omega + i\eta)I -H]G(\omega) = I
\end{equation}
\item The Self-Energy: From Newns-Anderson set or from the Green's function note using the EOM technique.
    \item Quasi-particle eigenvalue equation. At $\omega = \varepsilon_i^{QP}$ the LHS of \eqref{eq:QPEQ} diverges and hence the nominator must vanish.
\begin{equation}\label{eq:QPEQ}
    \sum_{i} \dfrac{\qty[\qty(\omega + i \eta) I - H^{0} - \Sigma\qty(t)] \ket{\Psi_i^{QP}}\bra{\Psi_i^{QP}}}{\omega - \varepsilon_i^{QP} + i \eta} = I 
\end{equation}
\item Self energy and approximations: \textbf{Wideband, Narrowband, (Elliptic)}.
\item Self energy changes from screening: Image charge.
\end{itemize}




\newpage
\section{Berry phase}
\begin{itemize}
    \item Parametric Hamiltonian
    \item Adiabatic limit
    \item Eigenstates (of any Hamiltonian) defined down to a gauge transformation.
    \item Finding the phase difference going from $R$ to $R+\mathrm{d}l$ is after linearisation
\begin{equation}
        \mathrm{d} \phi = i \braket{\psi_m(R)}{\pdv{R}\psi_m(R)} \cdot \mathrm{d}l
\end{equation}

\item This is the Berry connection.
\item Berry phase
\begin{equation}
    \gamma_n \qty(C) = i \oint_C \ip{n\qty(\vec{R})}{\nabla_{\vec{R}} n\qty(\vec{R})} \cdot \mathrm{d}\vec{R}
\end{equation}
\item Berry curvature
\begin{equation}
    \gamma_n = - \int \int_C \mathrm{d}\vec{S} \cdot \vec{B}_n\qty(\vec{R})
\end{equation}
\begin{equation}
    \vec{B}_n\qty(\vec{R}) = - \Im \sum_{m \neq m} \dfrac{\mel{n\qty(\vec{R})}{\nabla_{\vec{R}} \hat{H}\qty(\vec{R})}{m\qty(\vec{R})} \times \mel{m\qty(\vec{R})}{\nabla_{\vec{R}}H\qty(\vec{R})}{n\qty(\vec{R})}}{\qty(E_m\qty(\vec{R}) - E_n\qty(\vec{R}))^2}
\end{equation}
\item Connection to Aharanov-Bohm.
Make drawing
\item Barry Phase in solids: Chern number, Gauge patching.
\item Other topics: 
\end{itemize}
\end{document}